\documentclass{article}
\usepackage{graphicx}
\usepackage{enumerate}
\usepackage{setspace}
\usepackage{hyperref}
\usepackage[utf8]{inputenc}
\usepackage[OT1]{fontenc}
\usepackage[french]{babel}
\usepackage{vmargin}
\setpapersize{A4}
\setmargins{15mm}{15mm}{180mm}{260mm}{0pt}{0mm}{0pt}{1.2cm}
\title{Compte rendu OUTILS LIBRES}
\author{REVEL Rémi}
\date{\today}

\begin{document}
\maketitle
\par\noindent\rule{\textwidth}{0.4pt}

\section{\huge{efficatité de l'environnemnt de travail}}
%----------------------------------------Question 1-----------------------------------------------------
\subsection{\large{Desactivation de la souris}}

\subsection*{\normalsize{voici les commande pour desactiver la souris:} } 
\begin{enumerate}
    \item xinput set-prop 4 "Device Enabled" 0

    \item xinput set-prop 6 "Device Enabled" 0

    \item xinput set-prop 7 "Device Enabled" 0
\end{enumerate}

\subsection*{\normalsize{tableau de raccourcis clavier utile:} }    

\begin{center}
   \begin{tabular}{| l | c | }
     \hline
     changer d'application & alt+tab ou windows+tab\\ \hline
     gestionnaire d'application & Touche windows \\ \hline
     Naviguer sur les elements cliquable d'une page web & tab   \\ \hline
     Fermer le navigateur & CTRL+W \\ \hline
     Fermer une appliacation & ALT+F4 \\ \hline
     Changer d'onglet sous Brave & CTRL+1,2,3,...   \\ \hline
     Ouvrir un nouvelle onglet & CMD+T \\ \hline
     faire une recherche & F6 \\ \hline
   \end{tabular}
 \end{center}
 
 %---------------------------------------------------------------------------------------------
 
 %----------------------------------------question 2-----------------------------------------------------
 \subsection{\large{S'ameliorer a la dactylographie}}
 le site que j'ai retenu pour s'ameliorer en dactylographie est \href{https://10fastfingers.com/typing-test/french}{10fastfingers}. \par On peut s'entrainer sur des mots aleatoire, ou sur nos propres texte, le site est disponible dans plusieurs langue.

\begin{center}
    \includegraphics[scale=0.7]{Images/fastfinger.png}
\end{center}
%---------------------------------------------------------------------------------------------

%---------------------------------------------Question 3------------------------------------------------
\subsection{\large{Tutoriel pour VIM}}

\begin{center}
   \begin{tabular}{| l | c | }
     \hline
     insertion & i\\ \hline
     enregister & :w\\ \hline
     quitter & :q \\ \hline
     aller au debut du fichier & :1 \\ \hline
     aller a la fin du fichier & :\textdollar \\ \hline
     annuler une action & u \\ \hline
     recherche d'une occurence & /occurence \\ \hline
     activer coloration syntaxique & :syntax on \\ \hline
     templacer du texte & :s/origin/replacement/g \\ \hline
   \end{tabular}
 \end{center}
 
 pour definir VIM comme editeur par defaut on a juste a rentrer cette commande: \par update-alternatives --set editor /bin/vim
 %---------------------------------------------------------------------------------------------
 
 %---------------------------------------------Question 4------------------------------------------------
 \subsection{\large{Bash history}}
 
 Mon mot de passe n'apparait pas dans le bash history, donc il n'y a pas d'informations sensible.\par
 Les historiques sont propres a chaque shell utilisé. \\
 
 Pour eviter de poluer notre historique avec des commande basique on peut les exclures avec cette commande : \par
 export HISTIGNORE="ls : cd : pwd"
 
  %---------------------------------------------------------------------------------------------
  
  %---------------------------------------------Question 5------------------------------------------------
  \subsection{\large{Alias de fonction}}
  Les commandes presentes ci dessous sont a placer dans le .bashrc
  
    \begin{center}
        \includegraphics[scale=0.7]{Images/function.png}
    \end{center}
    
 %---------------------------------------------------------------------------------------------
 
 %---------------------------------------------Question 6------------------------------------------------
 \subsection{\large{Script}}
 Ce script est fait pour la sauvegarde des données des utilisateurs de la machine.
  \begin{center}
        \includegraphics[scale=0.5]{Images/save.png}
    \end{center}
    
 %---------------------------------------------------------------------------------------------
 \newpage
 %---------------------------------------------Question 7------------------------------------------------
 \subsection{\large{customisation avec OH MY ZSH}}
 Dans le fichier du theme oh y zsh que l'on a selectioné on y rajoute ceci pour pouvoir voir le status de nos Vagrant et notre statut git.
 
 \begin{center}
        \includegraphics[scale=0.43]{Images/theme.png}
    \end{center}
    
 %---------------------------------------------------------------------------------------------
 
 %---------------------------------------------Question 8------------------------------------------------
  \subsection{\large{Raccourci clavier}}
  
  Pour faire un raccourcis clavier qui start ou stop apache en faisant CTRL+A, on place ce petit bout de code dans notre .bashrc
  
   \begin{center}
        \includegraphics[scale=0.6]{Images/CTRLA.png}
    \end{center}
 %---------------------------------------------------------------------------------------------
 
 
\end{document}
